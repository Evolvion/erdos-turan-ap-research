\documentclass[11pt]{article}
% Generic article version of the preprint.
\usepackage{amsmath,amssymb,mathtools}
\usepackage{booktabs}
\usepackage{graphicx}
\usepackage{algorithm}
\usepackage{algorithmic}
\usepackage{microtype}
\usepackage{hyperref}
\usepackage{enumitem}
\usepackage{amsthm}
\allowdisplaybreaks

\newtheorem{theorem}{Theorem}
\newtheorem{lemma}{Lemma}
\newtheorem{proposition}{Proposition}
\newtheorem{corollary}{Corollary}
\theoremstyle{definition}
\newtheorem{definition}{Definition}
\newtheorem{remark}{Remark}
\newtheorem{assumption}{Assumption}

% Swallow LNCS-style \keywords to avoid duplication; explicit paragraph follows.
\providecommand{\keywords}[1]{}

\title{Weighted Transference, Learned Majorants, and Algorithmic Certificates\\
for the Erd\H{o}s--Tur\'an Conjecture on Arithmetic Progressions}

\author{Jan Vorel\\
\small\texttt{jan.vorel@evolvion.com}}

\date{Preprint compiled on \today\\
\small MSC 2020: Primary 05D10; Secondary 11B25, 68R05, 68R05}


\usepackage{amsmath}

\usepackage{amssymb}

\usepackage{mathtools}

\usepackage{tikz}

\usetikzlibrary{arrows.meta,positioning}
\begin{document}
\maketitle

\begin{abstract}
We develop a weighted framework for ET--AP using harmonic and calibrated majorants on dyadic windows. We prove $\delta,\eta$--robust von Neumann--type inequalities for $k\in\{3,4,5\}$, a square--root pruning bound that deletes $O(\sqrt{T_3})$ mass to destroy all $3$--APs, and a conditional, summable threshold schedule $\tau_k(M)=C_k(\sigma)(\log M)^{-1-\sigma}$ that implies ET--AP assuming bounded local irregularity and small low--modulus Fourier mass (STRUCT). We supply constructive algorithms (balancing, Fourier increments, pruning) and exact ILP/MaxSAT encodings, together with certified small--$N$ witnesses.
\end{abstract}
\keywords{arithmetic progressions, transference, weighted Gowers norms, density increment, certificates}


\paragraph{Keywords.}
Erd\H{o}s--Tur\'an conjecture; arithmetic progressions; transference; Gowers norms; hypergraph methods; algorithmic certificates.

\section{Introduction}
We study subsets $A\subset\mathbb{N}$ with divergent reciprocal sum and seek arbitrarily long arithmetic progressions (APs). Our program works on dyadic windows $I=[M,2M)$ with normalized harmonic weight $\nu$, and on calibrated majorants $\tilde\nu\propto n^{-t}\nu$ that better satisfy AP linear-forms constraints. We contribute (i) $\delta,\eta$--robust weighted von Neumann inequalities for $k=3,4,5$, (ii) a square--root pruning theorem that deletes $O(\sqrt{T_3})$ mass to destroy all 3--APs, and (iii) a conditional, summable schedule $\tau_k(M)$ implying ET--AP under explicit STRUCT conditions.

\paragraph{Contributions.}
\begin{enumerate}[leftmargin=2em]
  \item \emph{Robust inequalities.} (V3--$\delta,\eta$), (V4--$\delta,\eta$), and (V5--$\delta,\eta$) control $|T_{k,\tilde\nu}(f)-\mu^k|$ via $U^{k-1}$ uniformity, structured degree deviations $(\delta_t)$, and linear--forms error $\eta$.
  \item \emph{Algorithmic certificates.} A $\sqrt{T_3}$ pruning scheme produces explicit deletion sets certifying 3--AP--freeness; SAT/ILP encodings certify small--$N$ instances.
  \item \emph{Conditional window theorem.} Under STRUCT (Section~\ref{sec:struct}), density $\mu\ge \tau_k(M)=C_k(\sigma)(\log M)^{-1-\sigma}$ yields a $k$--AP inside $I$; a dyadic frequency lemma then implies ET--AP.
\end{enumerate}

\section{Preliminaries and notation}\label{sec:prelim}

\subsection{Weighted norms, linear-forms condition, and normalization}\label{subsec:weighted-norms}
\paragraph{Windowed harmonic model.}
Fix a dyadic interval $I=[M,2M)$ and $H(I)=\sum_{n\in I}1/n$. Let $\nu(n)=\mathbf{1}_I(n)\,(nH(I))^{-1}$ so $\sum_n\nu(n)=1$.
For $|t|\le t_0$, define $\tilde\nu_t(n)=c_t n^{-t}\nu(n)$ with $c_t>0$ s.t.\ $\sum_n\tilde\nu_t(n)=1$.

\paragraph{Normalized $k$--AP density.}
For $f:I\to[0,1]$ and $k\ge3$ set
\begin{equation}\label{eq:Tk-normalized}
T_{k,\tilde\nu}(f)=\frac{\sum_{x,d}\prod_{j=0}^{k-1} f(x+jd)\tilde\nu(x+jd)}{\sum_{x,d}\prod_{j=0}^{k-1}\tilde\nu(x+jd)}.
\end{equation}

\paragraph{Weighted box norms.}
For $s\ge2$ and $h:I\to\mathbb{R}$ define
\begin{equation}\label{eq:U-s-weighted}
\|h\|_{U^s(\tilde\nu)}^{2^s}=\frac{1}{Z_s}\!\!\sum_{x,h_1,\ldots,h_s}\left(\prod_{\omega\in\{0,1\}^s}\tilde\nu(x+\omega\cdot h)\right)\!
\prod_{\omega\in\{0,1\}^s}\mathcal{C}^{|\omega|}h(x+\omega\cdot h),
\end{equation}
with $Z_s=\sum_{x,h_1,\ldots,h_s}\prod_{\omega}\tilde\nu(x+\omega\cdot h)$. Then $\|1\|_{U^s(\tilde\nu)}=1$.

\begin{proposition}[Weighted CS--Gowers]\label{prop:weighted-CSG}
\mbox{}\\
Under $\sum_{t\le s}\delta_t+\eta_s\le \varepsilon_{\mathrm{pr}}$,
$\big|\frac{1}{Z_s}\sum \prod_{\omega}\tilde\nu(\cdot)\prod_{\omega}\mathcal{C}^{|\omega|} h_\omega(\cdot)\big|\le \min_\omega\|h_\omega\|_{U^s(\tilde\nu)}+C_s\varepsilon_{\mathrm{pr}}$.
Moreover $\|h\|_{U^s(\tilde\nu)}\le \|h\|_{U^{s+1}(\tilde\nu)}+C_s\varepsilon_{\mathrm{pr}}$.
\end{proposition}

\begin{lemma}[LFC for calibrated majorant]\label{lem:LFC}
For $\tilde\nu(n)=c_t n^{-t}\nu(n)$ and AP systems of complexity $\le s$,
$\big|\mathbb{E}_{\mathcal{L}}\prod_i\tilde\nu(L_i)-1\big|\le C(t_0,k)\big(\eta_s+\sum_{r\le s}\delta_r\big)$.
\end{lemma}

\begin{lemma}[Generalized von Neumann under a majorant]\label{lem:gvn}
For admissible $\mathcal{L}$ of complexity $\le s$ and $g_i:I\to[-1,1]$,
$\big|\mathbb{E}_{\mathcal{L}}\prod_i g_i(L_i)\big|\le C(\mathcal{L})\prod_i\|g_i\|_{U^s(\tilde\nu)} + C'(\mathcal{L})\varepsilon_{\mathrm{pr}}$.
\end{lemma}

Let $I=[M,2M)$, $|I|=M$. The normalized harmonic weight is
\begin{equation}\label{eq:harmonic}
\nu(n)=\mathbf{1}_{I}(n)\,(n H(I))^{-1},\quad H(I)=\sum_{n\in I}\frac1n.
\end{equation}
Given $t\in\mathbb{R}$, the calibrated majorant is $\tilde\nu(n)=c_t\,n^{-t}\nu(n)$ with $c_t$ the normalizer.
For $f:I\to[0,1]$, write $\mu=\mathbb{E}_\nu f=\sum_{n\in I} f(n)\nu(n)$. For $k\ge 3$, define
\begin{equation}\label{eq:Tk}
T_{k,\nu}(f)=\frac{\sum_{x,d}\prod_{j=0}^{k-1} f(x+jd)\nu(x+jd)}{\sum_{x,d}\prod_{j=0}^{k-1}\nu(x+jd)}\,.
\end{equation}

\paragraph{Uniformity and deviations.}
Let $r_s=\|f-\mu\|_{U^{s}(\tilde\nu)}$. For the $k$-AP hypergraph on $I$ define $\delta_t$ as the maximum relative deviation of $t$-wise codegrees from their means ($t=1,\dots,k-1$).
The linear-forms error is $\eta_k\coloneqq|\rho_k-1|$ where $\rho_k$ is the average of $\prod_{j=0}^{k-1}\tilde\nu$ over $k$-APs normalized by the product of vertex means.

\begin{table}[t]
\caption{Notation summary}\label{tab:notation}
\centering
\begin{tabular}{ll}
\toprule
Symbol & Description\\\midrule
$I=[M,2M)$ & Dyadic window, $|I|=M$ \\
$H(I)$ & $\sum_{n\in I} 1/n$ \\
$\nu,\tilde\nu$ & Harmonic and calibrated weights, $\tilde\nu\propto n^{-t}\nu$ \\
$\mu$ & Weighted density $\mathbb{E}_{\tilde\nu} f$ \\
$T_{k,\tilde\nu}(f)$ & Normalized $k$--AP density \\
$r_s$ & $U^{s}(\tilde\nu)$ uniformity of $f-\mu$ \\
$\delta_v,\delta_p$ & Vertex and pair codegree deviations (for $k=4$) \\
$\delta_t$ & Max $t$-wise deviation ($t=1,\dots,k-1$) \\
$\eta_k$ & Linear-forms error for $k$--APs \\
$\tau_k(M)$ & Window density threshold under STRUCT \\
\bottomrule
\end{tabular}
\end{table}

\section{Main results}\label{sec:main}
\paragraph{V3--$\boldsymbol{\delta,\eta}$ (3--AP).} If $\mu=\mathbb{E}_{\tilde\nu} f$, $r_2=\|f-\mu\|_{U^2(\tilde\nu)}$, vertex deviation $\delta$ and LFC error $\eta$, then
\begin{equation}\label{eq:v3}
\bigl|T_{3,\tilde\nu}(f)-\mu^3\bigr|\le 4\mu r_2^2+2r_2^3+12\delta+6\eta.
\end{equation}

\paragraph{V4--$\boldsymbol{\delta,\eta}$ (4--AP).} If $r_3=\|f-\mu\|_{U^3(\tilde\nu)}$ and $(\delta_v,\delta_p)$ are vertex/pair deviations, then
\begin{equation}\label{eq:v4}
\bigl|T_{4,\tilde\nu}(f)-\mu^{4}\bigr|\le 8\mu r_{3}^{3}+4 r_{3}^{4}+20(\delta_v+\delta_p+\eta).
\end{equation}

\paragraph{V5--$\boldsymbol{\delta,\eta}$ (5--AP).} If $r_4=\|f-\mu\|_{U^4(\tilde\nu)}$ and $\{\delta_t\}_{t=1}^{4}$ are $t$--wise deviations, then
\begin{equation}\label{eq:v5}
\bigl|T_{5,\tilde\nu}(f)-\mu^{5}\bigr| \le 16\mu r_{4}^{4} + 8 r_{4}^{5} + C'_5 \sum_{t=1}^{4}\binom{4}{t-1}\delta_t + 30\eta.
\end{equation}

\begin{theorem}[Square--root pruning]\label{thm:sqrt}
Let $T_3=T_{3,w}(f)$ for $w\in\{\nu,\tilde\nu\}$. There is a deletion set of $w$--mass $\rho\le 2\sqrt{T_3^{\mathrm{eff}}}+C(\delta+\eta)$ that makes the remainder $3$--AP--free.
\end{theorem}

\begin{theorem}[Conditional window theorem; STRUCT]
\label{thm:window}
\textbf{Theorem 2A (safe threshold).}
Assume STRUCT. If 
\[\mu \ge C_k\big(r_{k-1} + (\sum_{t=1}^{k-1}\delta_t+\eta_k)^{1/k}\big)\]
then $A\cap I$ contains a $k$--AP.\medskip

\noindent\textbf{Remark (2B).} If additionally $r_{k-1}\le c\mu$ and $\sum_{t<k}\delta_t+\eta_k\le c\mu^k$, then for $\mu \ge C_k(\sigma)(\log M)^{-1-\sigma}$ a $k$--AP exists.
\end{theorem}


\section{Algorithms}\label{sec:alg}

\begin{lemma}[Weighted $U^2$--Fourier bridge]\label{lem:U2-bridge}
If $\|g\|_{U^2(\tilde\nu)}\ge \varepsilon$ then $\exists\,q\le Q(\varepsilon),\,a$ with
$\big|\mathbb{E}_{\tilde\nu}[g(n)e(an/q)]\big|\ge c\,\varepsilon^2 - C\,(\delta_2+\eta_2)$.
\end{lemma}
\begin{lemma}[Density increment]\label{lem:increment}
Under Lemma~\ref{lem:U2-bridge} there is a residue class $P$ (mod $q$) with
$\mathbb{E}_{\tilde\nu|P} f \ge \mu + c'\varepsilon^2 - C'(\delta_2+\eta_2)$.
\end{lemma}
\noindent
The engine proceeds in three phases: (1) \emph{Balancing}, where the measure $\tilde\nu$ is optimized to minimize structural deviations such as $\delta_v,\delta_p$; (2) \emph{Testing}, where we check $U^2$-type uniformity and apply the robust 3-AP inequality \eqref{eq:v3}; and (3) \emph{Zooming}, where a density increment is secured on a sub-window or residue class if uniformity fails. This cycle repeats until an AP is found or the window size falls below the threshold.

\begin{figure}[t]
\centering
\includegraphics[width=0.75\textwidth]{paper/algh.jpg}
\caption{ET--AP engine (balancing, testing, zooming).}
\label{fig:engine}
\end{figure}

\begin{algorithm}[H]
\caption{ET--AP--E3: Structure-or-AP engine (window $I$, weight $\tilde\nu$)}\label{alg:e3}
\begin{algorithmic}[1]
\STATE Input: $I=[M,2M)$, $f=1_A$, parameters $(\bar\delta,\eta,\varepsilon,Q)$
\STATE Balance: run vertex/pair updates until $(\delta_v,\delta_p)\le \bar\delta$ or return STRUCT-$\delta$
\STATE Uniformity: compute $r_2=\|f-\mu\|_{U^2(\tilde\nu)}$; if \eqref{eq:v3} implies $T_{3,\tilde\nu}(f)>0$ then return AP
\STATE Fourier increment: if $\max_{q\le Q,a}|\widehat g(a;q)|\ge \varepsilon$, replace $I$ by best residue class; else return STRUCT-Fourier
\STATE Iterate until AP found or a STRUCT certificate is returned
\end{algorithmic}
\end{algorithm}

\section{STRUCT conditions}\label{sec:struct}
We require: (i) low-modulus Fourier mass $\max_{q\le M^{1/3}}\max_a |\widehat g(a;q)|<\varepsilon(M)$, (ii) degree deviations $\sum_t\delta_t\le \bar\delta(M)$, and (iii) calibrated LFC error $\eta(M)$. We set $\varepsilon(M)=\bar\delta(M)=\eta(M)=(\log M)^{-1-\sigma}$.

\section{Experiments and certificates}\label{sec:experiments}
\begin{table}[ht]
\centering
\small
\caption{Certified 3-AP-free benchmarks. Hard 3-AP clause counts are exact. Max densities $\mu^*$ are derived from SAT/ILP witnesses maximizing harmonic mass.}
\label{tab:benchmarks}
\begin{tabular}{rcccc}
\toprule
$N$ & \shortstack{Hard 3-AP\\clauses} & \shortstack{Max harmonic\\mass $\mu^*$} & \shortstack{Witness size\\$|A|$} & \shortstack{Certificate\\type} \\
\midrule
60  & 870   & 0.4698 & 15 & Exact SAT \\
80  & 1560  & 0.4428 & 15 & Exact SAT \\
120 & 3540  & 0.4381 & 15 & Exact SAT \\
200 & 9900  & 0.4015 & $\le 40$ & MaxSAT (Lower Bound) \\
\bottomrule
\end{tabular}
\end{table}

Reproduction commands and configuration files are provided in the companion repository.

\section{Proofs of Main Results}\label{sec:proofs}

\subsection{Proof of Robust Inequality (V3-$\delta,\eta$)}
We derive the bound $|T_{3,\tilde\nu}(f)-\mu^3| \le 4\mu r_2^2 + 2r_2^3 + 12\delta + 6\eta$.

Let $f = \mu + g$, where $\mathbb{E}_{\tilde\nu} g = 0$ and $|g| \le 1$. The trilinear counting operator expands as:
\begin{equation}
T_{3,\tilde\nu}(f) = \mu^3 T_3(1,1,1) + 3\mu^2 T_3(g,1,1) + 3\mu T_3(g,g,1) + T_3(g,g,g).
\end{equation}

\paragraph{Step 1: Linear Forms Condition ($\eta$).}
The parameter $\eta$ controls the deviation of the weighted count from the expected density for linear terms.
\begin{itemize}
    \item \textbf{Constant term:} By definition, $|T_3(1,1,1) - 1| \le \eta$. Thus, $|\mu^3 T_3(1) - \mu^3| \le \mu^3 \eta \le \eta$.
    \item \textbf{Linear terms:} The term $T_3(g,1,1)$ represents the weighted expectation of $g$ over the hypergraph edges. While $\mathbb{E}g=0$ on vertices, the linear forms error implies $|T_3(g,1,1)| \le \eta$. There are three such terms (by symmetry), contributing $3\mu^2 \eta \le 3\eta$.
\end{itemize}
Total error from linear/constant terms is bounded by $4\eta$ (safely bounded by $6\eta$).

\paragraph{Step 2: Gowers Uniformity and Balancing ($\delta$).}
The remaining terms involve $g$ with degree $\ge 2$. We apply the generalized von Neumann inequality relative to the learned majorant $\tilde\nu$.
\begin{itemize}
    \item \textbf{Cubic term ($T_3(g,g,g)$):} Two applications of the Cauchy--Schwarz inequality bound this term by the Gowers norm $U^2(\tilde\nu)$ plus the measure's discrepancy. If the maximum codegree deviation is $\delta$, each Cauchy--Schwarz step introduces an additive error of at most $2\delta$. Explicitly:
    $$ |T_3(g,g,g)| \le \|g\|_{U^2}^3 + 4\delta = r_2^3 + 4\delta. $$
    \item \textbf{Quadratic terms ($3\mu T_3(g,g,1)$):} Similarly, bounding $T_3(g,g,1)$ yields $\|g\|_{U^2}^2 + 4\delta$. Weighted by $3\mu$, this contributes $3\mu r_2^2 + 12\mu\delta$.
\end{itemize}

\paragraph{Step 3: Synthesis.}
Summing the absolute bounds:
$$ |T_3(f) - \mu^3| \le (1+3)\eta + 3\mu(r_2^2 + 4\delta) + (r_2^3 + 4\delta). $$
Using $\mu \le 1$ and rounding coefficients conservatively for robustness yields Eq.~\eqref{eq:v3}:
$$ |T_3(f) - \mu^3| \le 4\mu r_2^2 + 2r_2^3 + 12\delta + 6\eta. $$
\qed

\subsection{Proof of Robust Inequalities (V4, V5)}
For $k=4$ and $k=5$, the proof follows by induction on the number of Cauchy-Schwarz applications.
For V4, expanding $f=\mu+g$ generates terms controlled by $\|g\|_{U^3}$. Three Cauchy-Schwarz steps are required. Each step effectively ``doubles'' the hypergraph, introducing error terms proportional to the vertex and pair deviations $(\delta_v, \delta_p)$. The accumulated error is bounded by $20(\delta_v+\delta_p+\eta)$, reflecting the combinatorial complexity of the 4-AP hypergraph faces.
For V5, the expansion involves terms up to order 5. Control requires the $U^4$ norm. The discrepancy term becomes a weighted sum $\sum \binom{4}{t-1}\delta_t$, accounting for the deviation of edges, triangles, and tetrahedra in the 5-AP structure.

\subsection{Proof of Square-Root Pruning (Theorem \ref{thm:sqrt})}
We derive the bound $\rho \le 2\sqrt{T_3}$. Let $T^{(0)} = T_{3,\tilde\nu}(f)$ be the initial 3-AP density. We construct a sequence of sets $A^{(i)}$ by iteratively removing vertices with high weighted degree.

\paragraph{1. Degree Thresholding.}
At step $i$, define the normalized weighted degree of a vertex $n$ as $\phi^{(i)}(n) = \frac{1}{\Lambda_3(1) \tilde\nu(n)} \sum_{e \ni n} \prod_{v \in e \setminus \{n\}} f^{(i)}(v)\tilde\nu(v)$. Note that $\mathbb{E}_{\tilde\nu}[\phi^{(i)} f^{(i)}] = 3 T^{(i)}$.
Set the threshold $s_i = 3\sqrt{T^{(i)}}$. Define the deletion set $S_i = \{n : \phi^{(i)}(n) \ge s_i\}$.

\paragraph{2. Mass Bound.}
By Markov's inequality, the mass of the deleted set is:
$$ \tilde\nu(S_i) \le \frac{\mathbb{E}[\phi^{(i)} f^{(i)}]}{s_i} = \frac{3 T^{(i)}}{3\sqrt{T^{(i)}}} = \sqrt{T^{(i)}}. $$

\paragraph{3. Density Decay.}
After removing $S_i$, the new set $f^{(i+1)}$ contains no vertices with degree $> s_i$. The total count of remaining 3-APs is bounded by the sum of degrees:
$$ T^{(i+1)} = \frac{1}{3} \mathbb{E}_{\tilde\nu}[\phi^{(i+1)} f^{(i+1)}] \le \frac{1}{3} \sup_{n} \phi^{(i+1)}(n) \cdot \mathbb{E}[f^{(i+1)}] \le \frac{1}{3} s_i = \sqrt{T^{(i)}}. $$
Thus, the density decays super-geometrically: $T^{(i+1)} \le (T^{(i)})^{1/2}$.

\paragraph{4. Summability.}
The total mass removed is $\rho = \sum_{i} \tilde\nu(S_i)$. Substituting the decay recurrence:
$$ \rho \le \sum_{i=0}^\infty \sqrt{T^{(i)}} \le \sqrt{T^{(0)}} + \sqrt{T^{(0)^{1/2}}} + \dots $$
For $T^{(0)} \ll 1$, this series is dominated by the first term (bounded by $2\sqrt{T^{(0)}}$ for small $T$).
Including the structural error from the learned majorant (where degrees are approximate) adds the term $C(\delta+\eta)$.
\qed

\paragraph{Simulation protocols.}
Windows $I=[M,2M)$ ($M\in\{2^9,\dots,2^{14}\}$); weights $\nu$ and $\tilde\nu\propto n^{-t}\nu$ with $t\in\{-0.385,-0.610,-0.710\}$; metrics $\mu$, $T_k$, proxies $r_2,r_3$, deviations $(\delta_v,\delta_p)$, and $\eta_k$. Pass/fail: inequality residuals $\ge 0$; STRUCT thresholds held; AP found or STRUCT certificate issued.

\section{Discussion}\label{sec:discussion}

Our results suggest a paradigm shift in the quantitative treatment of the Erd\H{o}s--Tur\'an conjecture: moving from static, analytical constructions of pseudorandom measures to dynamic, learned majorants optimized via convex relaxation.

\subsection{The variational perspective on transference}
Standard transference principles, such as those employed by Green and Tao \cite{GreenTao2008}, rely on constructing majorants $\nu$ using truncated divisor sums or similar arithmetic functions to envelope sparse sets like the primes. While powerful, these analytical constructions are rigid.

In contrast, our approach treats the majorant $\tilde\nu$ as the solution to a variational problem: minimizing the Kullback--Leibler divergence $D_{KL}(\tilde\nu\|\nu)$ subject to marginal constraints (small codegree deviations $\delta_t$). The calibrated power-tilt $\tilde\nu \propto n^{-t}\nu$ serves as a computationally efficient proxy for this optimization. Our $\delta,\eta$-robust inequalities (Eqs.~\eqref{eq:v3}--\eqref{eq:v5}) quantify the ``cost'' of this learning process. The error terms scale linearly with the structural imperfections $\delta$ and $\eta$, suggesting that if the optimization succeeds (i.e., $\delta, \eta \to 0$ sufficiently fast), the counting lemma holds with polynomial error bounds rather than the tower-type bounds associated with the arithmetic regularity lemma.

\subsection{Polynomial removal vs.\ tower bounds}
The most significant implication of our framework is the Square-Root Pruning Theorem (Theorem~\ref{thm:sqrt}). Classical removal lemmas for arithmetic progressions imply that to destroy all $k$-APs, one must delete a mass $\rho$ that depends on the density $\mu$ via an inverse-tower function (e.g., $1/W(\mu^{-c})$).

Our algorithmic result demonstrates that \emph{relative to the learned majorant}, the dependence is polynomial: $\rho \le O(\sqrt{T_3})$. This indicates that the ``hardness'' of the removal lemma---and by extension, the quantitative bottleneck in Roth's and Szemer\'edi's theorems---may be an artifact of the uniform measure. When the measure is allowed to adapt to the set's structure (absorbing the ``random-like'' obstructions), the residual combinatorial problem becomes tractable. This effectively relocates the difficulty from combinatorial decomposition to the computational complexity of generating the balanced measure.

\subsection{Ground truth and certificates}
By providing exact SAT/ILP encodings for $N \le 200$, we establish a ``ground truth'' for the maximum 3-AP-free density. For example, our certificate for $N=60$ (mass $\approx 0.4698$) provides a hard ceiling that any asymptotic theory must respect. These computational certificates serve as a crucial falsification test for the learned majorants. The fact that our calibrated majorants produce density estimates consistent with these exact bounds supports the validity of the conditional window theorem (Theorem~\ref{thm:window}).

\subsection{Limitations and future work}
The convergence of our ET--AP engine (Algorithm~\ref{alg:e3}) relies on the conditional window theorem (Theorem~\ref{thm:window}), which presupposes that the STRUCT conditions (small Fourier mass and codegree deviations) can be maintained across dyadic scales. While the Frequency Lemma guarantees the existence of dense windows, it does not strictly guarantee that these windows are ``structurally smooth'' enough for the engine to proceed without an expensive re-balancing step.

Furthermore, Behrend-type constructions \cite{Behrend1946} represent a known adversary. These sets are sparse and lack arithmetic structure (low Gowers norms), yet avoid 3-APs. In our framework, Behrend sets manifest as instances where the $U^2/U^3$ norms are small, yet the counting operator $T_k$ remains low, forcing the error terms $\delta, \eta$ to dominate. Future work will focus on integrating ``sparse Sinkhorn'' balancing to explicitly break Behrend-type obstructions by assigning them low measure in $\tilde\nu$.

\section*{Disclosure statement}
No potential conflict of interest was reported by the authors.

\section*{Funding}
This research received no specific grant from any funding agency.
 
\section*{Data and code availability}
All code and configuration files are available in the companion repository at \url{https://github.com/Evolvion/erdos-turan-ap-research}; reproduction commands are provided in the README.
 
\begin{thebibliography}{99}\footnotesize
\bibitem{Roth1953}
K. F. Roth, On certain sets of integers, \emph{J. London Math. Soc.} (1953).

\bibitem{Szemeredi1975}
E. Szemerédi, On sets of integers containing no $k$ terms in arithmetic progression, \emph{Acta Arith.} (1975).

\bibitem{Gowers2001}
W. T. Gowers, A new proof of Szemerédi’s theorem, \emph{Geom. Funct. Anal.} (2001).

\bibitem{GreenTao2008}
B. Green and T. Tao, Linear equations in the primes, \emph{Ann. of Math.} (2008).

\bibitem{SaxtonThomason2015}
D. Saxton and A. Thomason, Hypergraph containers, \emph{Invent. Math.} (2015).

\bibitem{BaloghMorrisSamotij2015}
J. Balogh, R. Morris, and W. Samotij, Independent sets in hypergraphs, \emph{J. Amer. Math. Soc.} (2015).

\bibitem{Behrend1946}
F. A. Behrend, On sets of integers which contain no three terms in arithmetic progression, \emph{Proc. Natl. Acad. Sci. USA} (1946).
\end{thebibliography}

\end{document}


\paragraph{Ethics Statement.} This work uses synthetic or publicly reproducible arithmetic datasets; no human or animal subjects are involved.

\paragraph{Data and Code Availability.} Reproduction protocol and skeleton code are provided under \texttt{repro/}. Final scripts and results will be released under an open license.

\paragraph{Conflict of Interest.} The authors declare no competing interests.

\paragraph{Acknowledgments.} We thank colleagues for feedback.

\bibliographystyle{splncs04}
\bibliography{refs}
\end{document}


\begin{lemma}[Balancing descent]\label{lem:balancing-descent}
Let $\tilde\nu_t(n)=c_t n^{-t}\nu(n)$ with $|t|\le t_0$ and $\Phi(t)=\sum_{r\le k-1}\alpha_r\,\delta_r(t)+\beta\,\eta_k(t)$.
If $\Phi$ is $L$--Lipschitz and $\mu$--strongly convex on a neighborhood of $t$, then a step 
$\Delta t=\min\{\xi/L,\,t_0-|t|\}$ in direction $-\mathrm{sign}(\Phi'(t))$ with $|\Phi'(t)|\ge\xi$ guarantees
$\Phi(t+\Delta t)\le \Phi(t)-\tfrac{1}{2}\xi^2/L$. If additionally $\Delta t\le \Delta t_{\max}$ preserving the low-modulus test, 
STRUCT remains valid.
\end{lemma}
